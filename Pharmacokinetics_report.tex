\documentclass{jsarticle}

\begin{document}
\title{薬剤学実習\\薬物速度論}
\author{佐々木 優汰}
\maketitle
\tableofcontents

\section{目的}
吸収過程を伴う場合と、伴わない場合における1-コンパートメントモデルから実測値を得て薬物測度論パラメータを求めることで、血中薬物濃度を取り扱う薬物測度論の理解度を深める。

体内における薬物動態をコンパートメント(体の一部)モデル解析するための基礎的な薬物速度論的手法を習得する。

\section{方法}
\subsection{試薬}
\begin{itemize}
    \item 精製水入りの50mL遠沈管チューブ: 1
    \item ブリリアントブルーFCF溶液(100mg/L)入りの50mL遠沈管チューブ: 1
\end{itemize}
\subsection{器具}
\begin{itemize}
    \item ペリスタポンプ(シリコンチューブ付き): 1
    \item 台: 2
    \item 100mL三角フラスコ(自重が記載されている): 6
    \item 1L三角フラスコ: 1
    \item スターラー: 2
    \item 攪拌子: 2
    \item ピンチコック: 1
    \item 20mLメートルグラス:1
    \item 100mLまたは200mLメスシリンダー: 1
    \item ピペッター: 1
    \item 1mLピペット: 5
    \item セル: 1
    \item タイマー: 1
    \item ガラスのバイアル(サンプル入れ): 18
    \item シリコンチューブ: 短3, 長1(ペリスタポンプには使用しない)
    \item チューブ立て: 1
    \item 廃液入れガラス容器(「廃液」シール付き): 1
    \item ビニールテープ: 1
    \item サインペン: 2
    \item ハサミ: 1
    \item サンプリング容器: 2
    \item サービスタオル: 1
    \item ゴミ袋: 1
    \item 吸光光度計
\end{itemize}

\subsection{方法}

\subsubsection{実験1. 吸収過程を伴わない投与実験(静脈内投与を想定)}
\begin{itemize}
\setlength{\itemsep}{10mm}
\item 実験器具の準備
    \begin{enumerate}
        \item (図1)を参考に装置を組み立てた。(A)は体内(血中を想定)、(B)は排泄部位(尿を想定)に対応する。ピンチコックをチューブに通しておいた。
        \item (R)中に1L程度、(A)中に200mLの精製水を満たした。
        \item (A)の液面と(B)のチューブの開口部の高さがほぼ等しくなるように調整した。
        \item (A)の上蓋、およびサンプリング口のネジをしっかり締めた。
        \item (R)よりポンプ(ペリスタポンプ)を用いて精製水を送り込んだ。数分間、ポンプを動かし、(A)も水面の位置が変化しないことを確認した。(B)に入ってくる精製水の流速をメートルグラスとタイマーを用いて測定し、流速が10[mL/min]になるように調整した。終了後、(A), (B)は空にした。
    \end{enumerate}
\item 本題
    \begin{enumerate}
        \item サンプル瓶または三角フラスコにテープを貼り、ラベリングした。 
        \item 準備された約100mLのブリリアントブルーFCF溶液から、精製水で希釈し、約5mg/Lの溶液を200mL調整し、メスシリンダーとメートルグラスを用いて調整し、[A]に入れた。撹拌子も中に入れてスターラーを起動した。
        \item (A)より1mLを濃度測定用にサンプリングした。((A)の初濃度を得るため)
        \item (A)の上蓋およびサンプリング口のネジをしっかり締めた。
        \item (R)よりポンプを用いて精製水を送り込んだ。(ペリスタの回転の向きに注意)(A)に入り込む瞬間をt=0としてサンプリング口から5, 10, 20, 30, 40, 50分後にサンプルを1mLずつピペットで採取した。精製水を補充後(若干濃度が希釈されるが無視する)、ポンプのスイッチを入れ、実験再開した。
            \begin{itemize}
                \item サンプル採取方法
                    \begin{enumerate}
                        \item サンプル採取時間にポンプとタイマーを同時に止める。
                        \item (A)と(B)を繋ぐチューブをピンチコックで止める(サンプル採取可能)。
                        \item サンプリング口のネジを開け、サンプルを1mL採取し、ガラスのバイアルに入れる。その際、(A)への精製水1mLの補充をする。
                        \item サンプリング口のネジをしっかりと閉じた後、ピンチコックをはずし、ポンプとタイマーのスイッチを入れ、実験を再開する。この時、ピンチコックを外した状態で(A)を解放系にしないこと。
                    \end{enumerate}
            \end{itemize}
        \item 実験中、(A)の液面が変化しないことを確認した。変化するならば、閉鎖系でないのでやり直す。 
        \item 0-5, 5-10, 10-20, 20-30, 30-40, 40-50分の間隔で(B)に排泄される溶液を100mLの三角フラスコに回収した。(後に重量を測り液量を算出した)
        \item 手順5および手順7で採取したサンプルの630nmにおける吸光度を分光光度計で測定し、ブリリアントブルーFCF濃度を求めた。
        \item 
        \item 
        \item 
        \item 
    \end{enumerate}
\end{itemize}

\subsubsection{実験2. 吸収過程を伴う投与実験(経口投与を想定)}

\section{結果}

以下は箇条書きの例です。これは番号を振らない箇条書きです。

\begin{itemize}
  \item ちゃお
  \item りぼん
  \item なかよし
\end{itemize}

これは番号を振る箇条書きです。

\begin{enumerate}
  \item 富士
  \item 鷹
  \item なすび
\end{enumerate}

\section{考察}

これは一段組の例ですが,二段組に変更することもできます。

解説文を読んで,このソースをいろいろと変更してみましょう。

\section{感想}

\begin{thebibliography}{99}
  \item 奥村晴彦,黒木裕介『\LaTeXe 美文書作成入門』第7版(技術評論社,2017)
  \item ……
\end{thebibliography}
\end{document}
